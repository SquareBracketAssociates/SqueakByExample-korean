% $Author$
% $Date$
% $Revision$
%=================================================================
\ifx\wholebook\relax\else
% --------------------------------------------
% Lulu:
	\documentclass[a4paper,10pt,twoside]{book}
	\usepackage[
		papersize={6.13in,9.21in},
		hmargin={.815in,.815in},
		vmargin={.98in,.98in},
		ignoreheadfoot
	]{geometry}
	\input{../common.tex}
	\pagestyle{headings}
	\setboolean{lulu}{true}
% --------------------------------------------
% A4:
%	\documentclass[a4paper,11pt,twoside]{book}
%	\input{../common.tex}
%	\usepackage{a4wide}
% --------------------------------------------
    \graphicspath{{figures/} {../figures/}}
	\begin{document}
	\renewcommand{\nnbb}[2]{} % Disable editorial comments
	\sloppy
\fi
%=================================================================
\chapter{클래스와 메타클래스(Classes and metaclasses)}
\label{cha:metaclasses}

\on{The section on responsibilities of Class, Behavior and Metaclass are weak, and need to be fleshed out with convincing examples. Marcus, can you help?}

우리가 \charef{model}에서 보았듯이, \st에서는 모든 것이 오브젝트이며 모든 오브젝트는 클래스의 인스턴스입니다. 
클래스라고 예외는 아닙니다. 클래스도 오브젝트이며, 클래스 오브젝트는 다른 클래스의 인스턴스 입니다. 오브젝트 모델은 객체 지향 프로그래밍(object-oriented programming)의 정수를 캡쳐합니다: 이 모델은 군살이 없고 탄탄하며, 단순하고, 우아하며, 균일합니다. 
그럼에도 불구하고 이 균일성(uniformity)의 함의는 초보자에게 혼란을 줄 수 있습니다. 이 장의 목적은 ``매직(magic)''또는 여기서의 특별함(special)에 어떤 복잡한 것이 없다는 것을 보여드리는 일에 있습니다: 단지 단순한 규칙들이 일정하게 적용되었습니다. 
여러분은 이 규칙들을 따름으로써, 상황이 왜 그 방식으로 이루어졌는지를 항상 이해하실 수 있을 것입니다. 

%=================================================================
\section{클래스(class)와 메타클래스(metaclass)의 규칙들}

스몰토크 오브젝트 모델은 일정하게 적용된 제한된 수의 개념들에 기초합니다. 
스몰토크의 디자이너들은 Occam의 razor를 적용하였습니다: 필요한 것보다 좀더 복잡한 모델로 이끄는 모든 고려사항들은 폐기됩니다. 

여러분의 기억을 상기시켜드리기 위해, 우리가 \charef{model}에서 살펴본 오브젝트 모델들의 규칙들을 다음과 같이 기록하였습니다.  

\begin{enumerate}[label={\textbf{Rule \arabic{*}}.}, ref={Rule \arabic{*}}, leftmargin=*, widest=10]
% NB: rule labels must not be multiply defined across chapters!
\item{} % \label{rule:everything}
    모든 것은 오브젝트이다.

\item{} % \label{rule:instance}
    모든 오브젝트는 클래스의 인스턴스 이다.

\item{} % \label{rule:inheritance}
    모든 클래스는 수퍼클래스를 갖고 있다.

\item{} % \label{rule:message}
    모든 것은 메시지 발송에 의해 이루어 진다.

\item{} % \label{rule:lookup}
    메소드 검색(Method lookup)은 계층도 사슬(inheritance chanin)을 따른다.

\end{enumerate}

우리가 이 장의 도입 부분에 언급하였듯이 \ref{rule:everything}의 결과는 \emph{클래스 또한 오브젝트라는 것}이므로, \ref{rule:instance}는 우리에게 클래스가 반드시 클래스의 인스턴스가 되어야 한다고 말합니다. 
클래스의 클래스는 \emph{메타클래스(metaclass)}로 불립니다. 
\lable{sec:metaclassIntro}
\indmain{메타클래스}는 여러분이 클래스를 만들 때마다 여러분 자신을 위해 자동으로 만들어집니다. 
여러분은 대부분의 경우 메타클래스에 대해 신경을 쓸 필요가 없습니다. 
그럼에도 불구하고, 클래스의 ``class side''를 검색하기 위해 시스템 브라우저(system browser)를 사용할 때마다, 여러분이 실제로 다른 클래스(different class)를 검색하고 있으시다는 것을 상기하는 것이 도움이 됩니다. 
클래스와 그것의 메타클래스는, 비록 전자가 후자의 인스턴스임에도 불구하고 두 개의 분리된 클래스입니다.

클래스와 메타클래스를 적합하게 설명하기 위해, 우리는 \charef{model}에 나온 규칙들을 다음 추가 규칙들로 확장할 필요가 있습니다.

\begin{enumerate}[label={\textbf{Rule \arabic{*}}.}, ref={Rule \arabic{*}}, leftmargin=*, widest=10]
\setcounter{enumi}{5}
\item{} \label{rule:metaclass}
    모든 클래스는 메타클래스의 인스턴스이다.

\item{} \label{rule:parallelhierarchy}
    메타클래스 계층도(\subind{metaclass}{hierarchy})는 클래스 계층도와 평행을 이룬다.

\item{} \label{rule:behavior}
    모든 메타클래스는 \clsind{Class}와 \clsind{Behavior}를 상속한다.

\item{} \label{rule:metaclassclass}
    모든 메타클래스는 \clsind{Metaclass}의 인스턴스이다.

\item{} \label{rule:metaclassmetaclass}
	The metaclass of \ct{Metaclass} is an instance of \ct{Metaclass}.
    \ct{Metaclass}의 메타클래스는\ct{Metaclass}의 인스턴스이다.

\end{enumerate}

\noindent
이 10개의 규칙들은 함께 스몰토크의 오브젝트 모델을 완성합니다.

우리는 먼저, 작은 예시와 함께 \charef{model}에서 다룬 5개의 규칙들을 간략하게 다시 볼 것입니다. 그 다음 우리는 동일한 예시를 사용하여 새로운 규칙들을 면밀히 살펴볼 것입니다. 

%=================================================================
\section{스몰토크 오브젝트 모델을 다시 살펴보기}

모든 것이 오브젝트 이기 대문에 스몰토크에서 파란색 (blue) 색상 또한 오브젝트 입니다.
\begin{code}{@TEST}
Color blue --> Color blue
\end{code}

\noindent
모든 오브젝트는 클래스의 인스턴스입니다. Color blue의 클래스는 \clsind{Color}입니다. 
\begin{code}{@TEST}
Color blue class --> Color
\end{code}

\noindent
흥미롭게도, 색상의 알파 값(\emph{alpha})을 설정하면 \clsind{TranslucentColor}라는 다른 종류의 클래스의 인스턴스를 갖게됩니다. 
\begin{code}{@TEST}
(Color blue alpha: 0.4) class --> TranslucentColor
\end{code}

\noindent
모프(morph)를 만들고 그 색상을 반투명한 색상(translucent color)으로 설정할 수 있습니다.
\begin{code}{}
EllipseMorph new color: (Color blue alpha: 0.4); openInWorld
\end{code}
\noindent
여러분은 \figref{translucentEllipse}에서 그 효과를 볼 수 있습니다.

\begin{center}
\begin{figure}[!ht]
\ifluluelse
	{\centerline {\includegraphics[scale=0.7]{TranslucentEllipse}}}
	{\centerline {\includegraphics[width=8cm]{TranslucentEllipse}}}
\caption{반투명한 타원\label{fig:translucentEllipse}}
\end{figure}
\end{center}

\ref{rule:inheritance}에 의해, 모든 클래스는 superclass를 갖고 있습니다. 
\clsind{TranslucentColor}의 superclass는 \clsind{Color}이며, \ct{Color}의 superclass는 \clsind{Object}입니다:
\begin{code}{@TEST}
TranslucentColor superclass --> Color
Color superclass                   --> Object
\end{code}
모든 것은 메시지 발송(\ind{message send}{})에 의해(\ref{rule:message}) 이루어지므로, \mthind{Color class}{blue}는 \ct{Color}에 대한 메시지 이며, \mthind{Object}{class} 그리고 \mthind{Color}{alpha:}는 color blue에 대한 메시지 이며, \mthind{Morph}{openInWorld}는 타원 모프(ellipse morph)에 대한 메시지고, \mthind{Behavior}{superclass}는 \ct{TranslucentColor}와 \ct{Color}에 대한 메시지인 것을 추론할 수 있습니다. 
모든 것이 오브젝트이므로 각 사례의 수신자(receiver)는 오브젝트이지만, 동시에 이 오브젝트들의 몇몇은 클래스 이기도합니다.

메소드 보기(Mehod lookup)는 계층도 사슬(the inheritance chain, \ref{rule:lookup})을 따르며, 그러므로, 우리가 메시지 클래스(the message class)를 \ct{Color blue alpha: 0.4}의 결과에 발송하면, \figref{classmessage}에서 보이는 것 처럼, 클래스 오브젝트(the class object)에서 대응 메소드(the corresponding method)가 발견될 때, 메시지가 취급(handle)됩니다. 

\begin{center}
\begin{figure}[!ht]
\ifluluelse
	{\centerline{\includegraphics[width=\textwidth]{TranslucentClassMessage}}}
	{\centerline{\includegraphics[width=0.8\textwidth]{TranslucentClassMessage}}}
\caption{반투명 색상에 메시지 보내기 \label{fig:classmessage}}
\end{figure}
\end{center}

그림은 \emphind{is-a} 관계의 본질을 캡쳐합니다. 우리의 불투명 파랑(blue) 오브젝트는 TranslucentColor 인스턴스이지만, 우리는 또한 그것이 Color 이며, 오브젝트라고 말할 수 있고, 그 이유는 이 불투명 파랑 오브젝트는(blue object) 이 클래스들의 모든 것들에서 정의된 메시지들에 반응하기 때문입니다. 
사실, \mthind{Object}{isKindOf:} 메시지가 존재하며, 이 isKindOf: 메시지는 여러분이 주어진 클래스와 \emph{is a} 와의 관계 속에 이 isKindOf:가 있는지를 알아내기 위해 어떤 객체라도 발송할 수 있는 메시지입니다:

\begin{code}{@TEST | translucentBlue |}
translucentBlue := Color blue alpha: 0.4.
translucentBlue isKindOf: TranslucentColor --> true
translucentBlue isKindOf: Color                    --> true
translucentBlue isKindOf: Object                  --> true
\end{code}

%=================================================================
\section{Every class is an instance of a metaclass}

% \ruleref{metaclass}

As we mentioned in \secref{metaclassIntro}, classes whose instances are themselves classes are called \ind{metaclass}{}es.

\paragraph{Metaclasses are implicit.}
Metaclasses are automatically created when you define a class. We say that they are \emph{implicit} since as a programmer you never have to worry about them. An \subind{metaclass}{implicit} metaclass is created for each class you create, so each metaclass has only a single instance.
% At a more advanced level, this implies that sharing between metaclasses is difficult except by subclassing. 

Whereas ordinary classes are named by global variables, metaclasses are anonymous.
However, we can always refer to them through the class that is their instance.
The class of \clsind{Color}, for instance, is \clsind{Color class}, and the class of \ct{Object} is \clsind{Object class}:
\begin{code}{@TEST}
Color class   --> Color class
Object class --> Object class
\end{code}

\noindent
\figref{translucentmetaclasses} shows how each class is an instance of its (\subind{metaclass}{anonymous}) metaclass.

\begin{center}
\begin{figure}[!ht]
\ifluluelse
	{\centerline {\includegraphics[width=\textwidth]{TranslucentMetaclasses}}}
	{\centerline {\includegraphics[width=0.8\textwidth]{TranslucentMetaclasses}}}
\caption{The metaclasses of Translucent and its superclasses\label{fig:translucentmetaclasses}}
\end{figure}
\end{center}

The fact that classes are also objects makes it easy for us to query them by sending messages. Let's have a look:

\begin{code}{@TEST}
Color subclasses                           --> {TranslucentColor}
TranslucentColor subclasses         --> #()
TranslucentColor allSuperclasses  --> an OrderedCollection(Color Object ProtoObject)
TranslucentColor instVarNames     --> #('alpha')
TranslucentColor allInstVarNames --> #('rgb' 'cachedDepth' 'cachedBitPattern' 'alpha')
TranslucentColor selectors             --> an IdentitySet(#alpha: #asNontranslucentColor #privateAlpha #pixelValueForDepth: #isOpaque #isTranslucentColor #storeOn: #pixelWordForDepth: #scaledPixelValue32 #alpha #bitPatternForDepth: #hash #convertToCurrentVersion:refStream: #isTransparent #isTranslucent #setRgb:alpha: #balancedPatternForDepth: #storeArrayValuesOn:)
\end{code}
\cmindex{Class}{subclasses}
\cmindex{Behavior}{allSuperclasses}
\cmindex{Behavior}{instVarNames}
\cmindex{Behavior}{allInstVarNames}
\cmindex{Behavior}{selectors}

%=================================================================
\section{The metaclass hierarchy parallels the class hierarchy}
% \ruleref{parallelhierarchy}

\ref{rule:parallelhierarchy} says that the superclass of a metaclass cannot be an arbitrary class: it is constrained to be the metaclass of the superclass of the metaclass's unique instance.

\begin{code}{@TEST}
TranslucentColor class superclass --> Color class
TranslucentColor superclass class --> Color class
\end{code}

\noindent
This is what we mean by the metaclass \subindmain{metaclass}{hierarchy} being parallel to the class hierarchy;  \figref{parallelHierarchies} shows how this works in the \clsind{TranslucentColor} hierarchy.

\begin{center}
\begin{figure}[!ht]
\ifluluelse
	{\centerline {\includegraphics[width=\textwidth]{TranslucentMetaclassHierarchy}}}
	{\centerline {\includegraphics[width=0.8\textwidth]{TranslucentMetaclassHierarchy}}}
\caption{The metaclass hierarchy parallels the class hierarchy.\label{fig:parallelHierarchies}}
\end{figure}
\end{center}

\begin{code}{@TEST}
TranslucentColor class                                     --> TranslucentColor class
TranslucentColor class superclass                   --> Color class
TranslucentColor class superclass superclass --> Object class
\end{code}

\paragraph{Uniformity between Classes and Objects.}
It is interesting to step back a moment and realize that there is no difference between sending a message to an object and to a class. 
In both cases the search for the corresponding method starts in the class of the receiver, and proceeds up the inheritance chain.

Thus, messages sent to classes must follow the metaclass inheritance chain.
Consider, for example, the method \mthind{Color class}{blue}, which is implemented on the \subind{system browser}{class side} of \ct{Color}.
If we send the message \ct{blue} to \ct{TranslucentColor}, then it will be looked-up the same way as any other message.
The lookup starts in \ct{TranslucentColor class}, and proceeds up the metaclass hierarchy until it is found in \ct{Color class} (see \figref{metaclasslookup}).

\begin{code}{@TEST}
TranslucentColor blue --> Color blue
\end{code}
\noindent
Note that we get as a result an ordinary \ct{Color blue}, and not a translucent one\,---\,there is no magic!

\begin{center}
\begin{figure}[!ht]
\ifluluelse
	{\centerline {\includegraphics[width=\textwidth]{TranslucentColorBlue}}}
	{\centerline {\includegraphics[width=0.8\textwidth]{TranslucentColorBlue}}}
\caption{Message lookup for classes is the same as for ordinary objects.\label{fig:metaclasslookup}}
\end{figure}
\end{center}

Thus we see that there is one uniform kind of method \subind{method}{lookup} in Smalltalk. Classes are just objects, and behave like any other objects. 
Classes have the power to create new instances only because classes happen to respond to the message \ct{new}, and because the method for \ct{new} knows how to create new instances.
Normally, non-class objects do not understand this message, but if you have a good reason to do so, there is nothing stopping you from adding a \ct{new} method to a non-metaclass.

Since classes are objects, we can also inspect them.
\index{inspector}

\dothis{Inspect \ct{Color blue} and \ct{Color}.}

\noindent
Notice that in one case you are inspecting an instance of \ct{Color} and in the other case the \ct{Color} class itself.  
This can be a bit confusing, because the title bar of the inspector names the \emph{class} of the object being inspected.

The inspector on \ct{Color} allows you to see the superclass, instance variables, method \subind{method}{dictionary}, and so on, of the \ct{Color} class, as shown in \figref{inspectingColor}.

\begin{center}
\begin{figure}[!ht]
\ifluluelse
	{\centerline{\includegraphics[width=\textwidth]{InspectingColor}}}
	{\centerline{\includegraphics[width=10cm]{InspectingColor}}}
\caption{Classes are objects too.\label{fig:inspectingColor}}
\end{figure}
\end{center}

%=================================================================
\section{Every metaclass Inherits from \lct{Class} and \lct{Behavior}}
% \ruleref{behavior}

Every \ind{metaclass} \emphind{is-a} class, hence inherits from \clsind{Class}.
\ct{Class} in turn inherits from its superclasses, \clsind{ClassDescription} and \clsind{Behavior}.
Since everything in \st \emph{is-an} object, these classes all inherit eventually from \ct{Object}.
We can see the complete picture in \figref{inheritbehavior}.

\begin{center}
\begin{figure}
\ifluluelse
	{\centerline{\includegraphics[width=\textwidth]{TranslucentBehavior}}}
	{\centerline{\includegraphics[width=0.8\textwidth]{TranslucentBehavior}}}
\caption{Metaclasses inherit from \lct{Class} and \lct{Behavior}\label{fig:inheritbehavior}}
\end{figure}
\end{center}

\paragraph{Where is \lct{new} defined?}
To understand the importance of the fact that metaclasses inherit from \ct{Class} and \ct{Behavior}, it helps to ask where \ct{new} is defined and how it is found. When the message \ct{new} is sent to a class it is looked up in its metaclass chain and ultimately in its superclasses \ct{Class}, \ct{ClassDescription} and \ct{Behavior} as shown in \figref{sendingnew}.

The question \emph{``Where \ct{new} is defined?''} is crucial. \mthind{Behavior}{new} is first defined in the class \ct{Behavior}, and it can be redefined in its subclasses, including any of the metaclass of the classes we define, when this is necessary. Now when a message \ct{new} is sent to a class it is looked up, as usual, in the metaclass of this class, continuing up the superclass chain right up to the class \ct{Behavior}, if it has not been redefined along the way.

Note that the result of sending \ct{TranslucentColor new} is an instance of \clsind{TranslucentColor} and \emph{not} of \ct{Behavior}, even though the method is looked-up in the class \ct{Behavior}!  \ct{new} always returns an instance of \self, the class that receives the message, even if it is implemented in another class.

\begin{code}{@TEST}
TranslucentColor new class --> TranslucentColor    "not Behavior!"
\end{code}

\begin{center}
\begin{figure}
\ifluluelse
	{\centerline{\includegraphics[width=\textwidth]{TranslucentSendingNew}}}
	{\centerline{\includegraphics[width=0.8\textwidth]{TranslucentSendingNew}}}
\caption{\lct{new} is an ordinary message looked up in the metaclass chain.\label{fig:sendingnew}}
\end{figure}
\end{center}

A common mistake is to look for \ct{new} in the superclass of the receiving class. The same holds for \ct{new:}, the standard message to create an object of a given size.
For example, \lct{Array new: 4} creates an array of 4 elements.
You will not find this method defined in \ct{Array} or any of its superclasses.
Instead you should look in \ct{Array class} and its superclasses, since that is where the lookup will start.

\on{The text below needs more details and convincing examples ...}

\paragraph{Responsibilities of \lct{Behavior}, \lct{ClassDescription} and \lct{Class}.}
\clsind{Behavior} provides the minimum state necessary for objects that have instances: this includes a superclass link, a method dictionary, and a description of the instances (\ie representation and number).
\on{not sure I understand the last point}
\ct{Behavior} inherits from \ct{Object}, so it, and all of its subclasses, can behave like objects. 

\ct{Behavior} is also the basic interface to the compiler.
It provides methods for creating a method dictionary,
compiling methods,
creating instances (\ie \mthind{Behavior}{new}, \mthind{Behavior}{basicNew}, \mthind{Behavior}{new:}, and \mthind{Behavior}{basicNew:}),
manipulating the class hierarchy (\ie \mthind{Behavior}{superclass:}, \mthind{Behavior}{addSubclass:}),
accessing methods (\ie \mthind{Behavior}{selectors}, \lmthind{Behavior}{allSelectors}, \mthind{Behavior}{compiledMethodAt:}),
accessing instances and variables (\ie \mthind{Behavior}{allInstances}, \mthind{Behavior}{instVarNames}\ldots),
accessing the class hierarchy (\ie \mthind{Behavior}{superclass}, \mthind{Behavior}{subclasses})
and querying (\ie \mthind{Behavior}{hasMethods}, \mthind{Behavior}{includesSelector}, \mthind{Behavior}{canUnderstand:}, \mthind{Behavior}{inheritsFrom:}, \mthind{Behavior}{isVariable}).


\clsind{ClassDescription} is an abstract class that provides facilities needed by its two direct subclasses, \clsind{Class} and \clsind{Metaclass}.
\clsind{ClassDescription} adds a number of facilities to the basis provided by \ct{Behavior}:
named instance variables,
the categorization of methods into protocols,
the notion of a name (abstract),
the maintenance of change sets and the logging of changes, and
most of the mechanisms needed for filing-out changes.

\clsind{Class} represents the common behaviour of all classes.
It provides a class name, compilation methods, method storage, and instance variables.
It provides a concrete representation for class variable names and shared pool variables (\mthind{Class}{addClassVarName:}, \mthind{Class}{addSharedPool:}, \mthind{Class}{initialize}).
\ct{Class} knows how to create instances, so all metaclasses should inherit ultimately from \ct{Class}.

%=================================================================
\section{Every metaclass is an instance of \lct{Metaclass}}
% \ruleref{metaclassclass}

Metaclasses are objects too; they are instances of the class \clsind{Metaclass} as shown in \figref{metaclassclass}. The instances of class \ct{Metaclass} are the anonymous metaclasses, each of which 
has exactly one instance, which is a class.

\begin{center}
\begin{figure}
\ifluluelse
	{\centerline{\includegraphics[width=\textwidth]{TranslucentMetaclassClass}}}
	{\centerline{\includegraphics[width=0.8\textwidth]{TranslucentMetaclassClass}}}
\caption{Every metaclass is a \lct{Metaclass}.\label{fig:metaclassclass}}
\end{figure}
\end{center}

\ct{Metaclass} represents common metaclass behaviour.
It provides methods for instance creation (\lct{sub\-class\-Of:})
creating initialized instances of the metaclass's sole instance,
initialization of class variables,
metaclass instance,
% (\ct{name:inEnvironment:subclassOf:})
% Actually, this is in ClassBuilder, it seems!
method compilation, % (different semantics can be introduced)
and class information (inheritance links, instance variables, \etc).
\ab{This is too cryptic for me.}

%=================================================================
\section{The metaclass of \lct{Metaclass} is an Instance of \lct{Metaclass}}
% \ruleref{metaclassmetaclass}

The final question to be answered is: what is the class of \clsind{Metaclass class}?

The answer is simple: it is a metaclass, so it must be an instance of \ct{Metaclass}, just like all the other metaclasses in the system (see \figref{metaclassclassclass}).

\begin{center}
\begin{figure}
\ifluluelse
	{\centerline{\includegraphics[width=\textwidth]{TranslucentMetaclassClassClass}}}
	{\centerline{\includegraphics[width=0.8\textwidth]{TranslucentMetaclassClassClass}}}
\caption{All metaclasses are instances of the class \lct{Metaclass}, even the metaclass of \lct{Metaclass}.\label{fig:metaclassclassclass}}
\end{figure}
\end{center}

The figure shows how all metaclasses are instances of \ct{Metaclass}, including the metaclass of \ct{Metaclass} itself.
If you compare Figures \ref{fig:metaclassclass} and \ref{fig:metaclassclassclass} you will see how the metaclass \subind{metaclass}{hierarchy} perfectly mirrors the class hierarchy, all the way up to \ct{Object class}.

The following examples show us how we can query the class hierarchy to demonstrate that \figref{metaclassclassclass} is correct.
(Actually, you will see that we told a white lie\,---\,\ct{Object class superclass -->} {\clsind{ProtoObject class}, not \ct{Class}. In \sq, we must go one superclass higher to reach \ct{Class}.)

\begin{example}{The class hierarchy}{@TEST}
TranslucentColor superclass --> Color
Color superclass                   --> Object
\end{example}

\needlines{4}
\begin{example}{The parallel metaclass hierarchy}{@TEST}
TranslucentColor class superclass   --> Color class
Color class superclass                     --> Object class
Object class superclass superclass --> Class    "NB: skip ProtoObject class"
Class superclass                              --> ClassDescription
ClassDescription superclass            --> Behavior
Behavior superclass                         --> Object
\end{example}

\begin{example}{Instances of Metaclass}{@TEST}
TranslucentColor class class --> Metaclass
Color class class                   --> Metaclass
Object class class                 --> Metaclass
Behavior class class              --> Metaclass
\end{example}
\begin{example}{Metaclass class is a Metaclass}{@TEST}
Metaclass class class --> Metaclass
Metaclass superclass --> ClassDescription
\end{example}

%=================================================================
\section{Chapter summary}
Now you should understand better how classes are organized and the impact of a uniform object model. If you get lost or confused, you should always remember that message passing is the key: you look for the method in the class of the receiver. 
This works on \emph{any} receiver. 
If the method is not found in the class of the receiver, it is looked up in its superclasses.

\begin{itemize}
\item Every class is an instance of a metaclass.
	Metaclasses are implicit. A Metaclass is created automatically when you create the class that is its sole instance.

\item The metaclass hierarchy parallels the class hierarchy.
	Method lookup for classes parallels method lookup for ordinary objects, and follows the metaclass's superclass chain.

\item Every metaclass inherits from \ct{Class} and \ct{Behavior}.
	Every class \emph{is a} \ct{Class}. Since metaclasses are classes too, they must also inherit from \ct{Class}.
	\ct{Behavior} provides behaviour common to all entities that have instances.

\item Every metaclass is an instance of \ct{Metaclass}.
	\ct{ClassDescription} provides everything that is common to \ct{Class} and \ct{Metaclass}.

\item The metaclass of \ct{Metaclass} is an instance of \ct{Metaclass}.
	The \emph{instance-of} relation forms a closed loop, so \ct{Metaclass class class --> Metaclass}.
\end{itemize}
%=================================================================
\ifx\wholebook\relax\else\end{document}\fi
%=================================================================

%-----------------------------------------------------------------

